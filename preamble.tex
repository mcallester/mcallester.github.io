\usepackage{amsmath,amssymb,amsthm,graphicx,color}

\newcommand{\slidetwo}[2]{\vfill
\vfill
\centerline{\Large\thepage}
\eject
\centerline{\bf #1}
\bigskip
\centerline{\bf #2}
\vfill}

\newcommand{\slidetwoplain}[2]{\vfill
\eject
\centerline{\bf #1}
\bigskip
\centerline{\bf #2}
\vfill}

\newcommand{\slidethree}[3]{\vfill
\eject
\centerline{\bf #1}
\centerline{\bf #2}
\centerline{\bf #3}
\vfill}

\newcommand{\slide}[1]{
  \vfill
  \centerline{\Large\thepage}
  \eject
  \centerline{\bf #1}
  \vfill}

\newcommand{\slideplain}[1]{
  \vfill
  \eject
  \centerline{\bf #1}
  \vfill}

\newcommand{\anaslide}[1]{
  \vfill
  \centerline{\Large\thepage}
  \eject \centerline{\bf #1}}

\newcommand{\anaslideplain}[1]{
  \vfill
  \eject
  \centerline{\bf #1}}

\newcommand{\bigsum}[2]{\mathop{{\huge \Sigma}}_{#1}^{#2}\;}

\newcommand{\tail}[1]{\Phi\left(#1\right)}

\newcommand{\Rmax}{R_{\max}}
\newcommand{\dmax}{d_{\max}}
\newcommand{\emax}{e_{\max}}
\newcommand{\domx}{{\cal X}}
\newcommand{\domy}{{\cal Y}}
\newcommand{\domr}{{\cal R}}

\newcommand{\var}[1]{\mbox{\tt{"}}#1\mbox{\tt{"}}}
\newcommand{\phrase}[1]{{\mathrm Phrase}(#1)}
\newcommand{\branch}[1]{{\mathrm Branch}(#1)}
\newcommand{\terminal}[1]{{\mathrm Terminal}(#1)}

\newcommand{\mtt}[1]{\mbox{\tt #1}}
\newtheorem{theorem}{\noindent Theorem}
\newtheorem{lemma}[theorem]{\noindent Lemma}
\newtheorem{observation}[theorem]{\noindent Observation}
\newtheorem{corollary}[theorem]{\noindent Corollary}
\newtheorem{conjecture}[theorem]{\noindent Conjecture}
\newtheorem{proposition}[theorem]{\noindent Proposition}
\newtheorem{example}{\noindent Example}
\newtheorem{definition}{\noindent Definition}
\newtheorem{claim}[theorem]{\noindent Claim}
\newtheorem{fact}[theorem]{\noindent Fact}

\DeclareMathOperator*{\argmax}{argmax}
\DeclareMathOperator*{\argmin}{argmin}
\DeclareMathOperator*{\kbest}{kbest}
\DeclareMathOperator*{\softmax}{softmax}
\DeclareMathOperator*{\expsoftmax}{expsoftmax}

\DeclareMathOperator*{\locmax}{locmax}
\DeclareMathOperator*{\locmin}{locmin}

\newcommand{\parens}[1]{\left(#1\right)}
\newcommand{\brackets}[1]{\left[#1\right]}
\newcommand{\nn}{\nonumber \\}

\newcommand{\expect}[1]{\mathrm{E}\left[#1\right]}
\newcommand{\expectsub}[2]{E_{#1}\;#2}
\newcommand{\probsub}[2]{\mathrm{P}_{#1}\left[#2\right]}

\newcommand{\tuple}[1]{{\mbox{$\langle#1\rangle$}}}

\newcommand{\floor}[1]{\left\lfloor #1 \right\rfloor}

\newcommand{\xmean}{\expect{X}}
\newcommand{\xxmean}{\overline{x}}
\newcommand{\sbound}{\tilde{\sigma}}
\newcommand{\betamin}{\beta_{\min}}
\newcommand{\betamax}{\beta_{\max}}
\newcommand{\xmin}{x_{\min}}
\newcommand{\xmax}{x_{\max}}
\newcommand{\expectbeta}[1]{\expectsub{\beta}{#1}}

\newcommand{\calx}{{\cal X}}
\newcommand{\caly}{{\cal Y}}

\newcommand{\dint}{\int\!\!\!\!\int}
\newcommand{\classcount}[1]{\mathrm{c}(#1)}
\newcommand{\smallprod}{{\prod}}

\newcommand{\ignore}[1]{}

\newcommand{\weight}[1]{{\mathrm Weight}(#1)}
\newcommand{\context}[1]{{\mathrm Context}(#1)}
\newcommand{\econtext}[1]{{\mathrm EContext}(#1)}
\newcommand{\sstop}{{\mathrm stop}}
\newcommand{\sstart}{{\mathrm start}}

\newcommand{\sidebyside}[2]{\parbox[t]{2.0in}{#1}\hspace{1.0in}
                            \parbox[t]{2.0in}{#2}}

\newcommand{\tightsidebyside}[2]{\parbox[t]{2.0in}{#1}\hspace{.3in}
                            \parbox[t]{2.0in}{#2}}

\newcommand{\subproves}[1]{\,\;\vdash\!\!_{#1}\;\,}

\def\ant#1{\\ \> $#1$}

\def\unnamed#1#2{
\parbox{2.0in}{
\begin{tabbing}
\hspace{1em}\= #1 \\ \> \parbox{.75in}{\noindent \hrule ~} #2
\end{tabbing}
}}

\newcommand{\type}{\mathbf{Type}}

\newcommand{\abs}{\text{\emph{abs}}}

\newcommand{\reals}{\mathbb{R}}
\newcommand{\complex}{\mathbb{C}}

\newcommand{\grad}{\mathrm{grad}}
\newcommand{\pop}{\mathrm{Pop}}
\newcommand{\train}{\mathrm{Train}}
\newcommand{\popd}{\mathrm{pop}}
\newcommand{\rmvalue}{\mathrm{value}}
\newcommand{\kl}{\mathrm{KL}}

\newcommand{\pluseq}{\mbox{\tt +=}}

\newcommand{\minuseq}{\mbox{\tt -=}}

\newcommand{\cev}[1]{\reflectbox{\ensuremath{\vec{\reflectbox{\ensuremath{#1}}}}}}
\newcommand{\dvec}[1]{\stackrel{\leftrightarrow}{#1}}
\newcommand{\lmax}{L_{\max}}

\newcommand{\bbone}{\mathbf{1}}

\newcommand{\intype}[2]{#1\!:\!#2}
\def\dcolon{\mathrel{:}\joinrel\mathrel{\mkern+5mu}\joinrel\mathrel{:}}
\newcommand{\inntype}[2]{#1\dcolon#2}
\newcommand{\bool}{\mathrm{\bf Bool}}

\newcommand{\relu}{\mathrm{ReLU}}
\newcommand{\conv}{\mathrm{CONV}}
\newcommand{\pad}{\mathrm{pad}}
\newcommand{\stride}{\mathrm{stride}}

\newcommand{\images}{../images}

