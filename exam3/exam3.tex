\documentclass{article}

\input ../preamble.tex

\usepackage{amsmath,amssymb,amsthm,graphicx,color}

\parindent = 0em
\parskip = 2ex

\newcommand{\solution}[1]{}
%\newcommand{\solution}[1]{\bigskip {\color{red} {\bf Solution}: #1}}

\begin{document}


\centerline{\bf TTIC 31230 Fundamentals of Deep Learning, Autumn 2021}
\bigskip
\centerline{\bf Exam 3}


















\bigskip
{\bf Problem 2. 25 pts}  This problem is on GAN language modeling.  A GAN takes noise as input and transforms it to an output.  We consider the case where the output is a string of symbols $w_1,\ldots, w_T$
where for simplicity we always generate a string of exactly length $T$ and where the words are integers with $w_t \in \{0,\ldots,I-1\}$ where $I$ is the size of the vocabulary.
The GAN parameters are just the parameters of a bigram model, i.e., the parameters are probability tables

\begin{eqnarray*}
  P[i] & = & P(w_1 = i) \\
  \\
  Q[i,j] & = & P(w_{t+1} = j\;|\; w_t = i)
\end{eqnarray*}

We take the noise input to the GAN to be a sequence of random real numbers $\epsilon_1,\ldots,\epsilon_T$ where each $\epsilon_t$ is drawn uniformly from the interval $[0,1]$.

{\bf (a)} Write a function $\hat{w}(P[I],\epsilon_1)$ which deterministically returns the first word given the noise value $\epsilon_1$ such that the probability over the draw of $\epsilon_1$
that $\hat{w}(P[I],\epsilon_1) = i$ is $P[i]$.

\solution{
  We can take $\hat{w}(P[I],\epsilon_1)$ to be the unique $i$ such that $\epsilon_1 \in \left[\left(\sum_{j<i} P[j]\right),\;\left(\sum_{j \leq i} \;P[j]\right)\right]$
}

{\bf (b)} Write a function $\hat{w}(Q[I,I],w_t,\epsilon_t)$ which deterministically returns the word $w_{t+1}$ given $w_t$ such that the probability over the draw of $\epsilon_t$
that $\hat{w}(Q[I,I],w_t,\epsilon_t) = j$ is $Q[w_t,j]$.

\solution{
  We can take $\hat{w}(Q[I,I],w_t,\epsilon_t)$ to be the unique $w_j$ such that $\epsilon_t \in \left[\left(\sum_{j<i} Q[w_t,j]\right),\;\left(\sum_{k \leq j} \;Q[w_t,j]\right)\right]$
}

{\bf (c)} There is a problem with this GAN.  For string generated by the GAN we need to back-propagate the discriminator loss into the GAN generator parameters.  Explain why this is problematic.
Is this always problematic when the generator output is discrete?

\solution{Yes, there is a problem whever $s$ is discrete. A discrete output will not change under differential updates to the GAN parameters.  Hence the gradient of the discriminator loss
  with respect to the generator parameters is zero.  This will happen for any GAN generatng a discrete output. While there are approaches one can try for discrete GANs, GANs are most effective for modeling
  continuous objects like sounds and images.  It does not help to have the GAN sample from a transformer model. To get a gradient on the generator parameters we need a gradient of the discriminator loss with
  respect to a continuous signal $s$ being generated by the generator.}

\bigskip
{\bf Problem 3. 25 pts} This problem is on VAE language modeling (in contrast to GAN language modeling).  Consider a VAE where the signal $s$ is a word string $w_1,\ldots,w_T$ (as in problem 2).  In the VAE 
we can have a continuous latent variable $z$. The VAE optimization problem is then
\begin{equation}
    \Phi^*,\Theta^*,\Psi^* = \argmin_{\Phi,\Theta,\Psi}\;E_{s \sim \pop,\;z \sim p_\Psi(z|s)}\;\ln\frac{p_\Psi(z|s)}{p_\Phi(z)} \;-\; \ln P_\Theta(s|z)
    \label{VAE}
\end{equation}
Here the first ``rate term'' is defined on densities and the final ``distortion term'' is defined for a discrete sentence $s$.
To explicitly handle the reparameterization trick will take the encoder density to be a Gaussian.
For a Gaussian encoder we compute a mean vector $\hat{z}_\Psi(s)$ and a variance $\sigma^2_\Psi(s)[i]$ for each component
$z[i]$ of $z$.  The Gaussian density for the encoder is then.
$$ p_\Psi(z[i]|s) \propto \exp(-(z[i]-\hat{z}_\Psi(s)[i])^2/(2\sigma^2_\Psi(s)[i])$$

{\bf (a)} For a noise value $\epsilon \in \reals$ drawn from ${\cal N}(0,1)$,
and for given values $\hat{z} \in \reals$ and $\sigma^2 \in \reals$, define a deterministic function $z(\hat{z},\sigma^2,\epsilon)$ such that over the draw of the noise
$\epsilon$ we have that $z(\hat{z},\sigma^2,\epsilon)$ has the density
$$p(z) \propto \exp(-(z-\hat{z})^2/(2\sigma^2)).$$

\solution{$z(\hat{z},\sigma^2,\epsilon) = \hat{z} + \sigma\epsilon$}

{\bf (b)} Applying your solution to part (a) to the individual components of $z$ equation~(\ref{VAE}) can be rewritten as
\begin{equation}
    \Phi^*,\Theta^*,\Psi^* = \argmin_{\Phi,\Theta,\Psi}\;E_{s \sim \pop,\;\epsilon \sim {\cal N}(0,I)}\;\ln\frac{p_\Psi(z|s)}{p_\Phi(z)} \;-\; \ln P_\Theta(s|z)
    \label{VAE2}
\end{equation}
Are there any problems with doing SGD on the optimization defined by (\ref{VAE2}) due to the use of continuous $z$ and discrete $s$?  Explain your answer.

\solution{There are no problems here.
  Since $P(s|z)$ is a computable and $z$ is continuous we can compute $z.\grad$ which can then passed back to the encoder $\Psi$
  through the computation of $z(\hat{z}_\Psi(y),\Sigma_\Psi(y),\epsilon)$. We get a clear advantage of VAEs over GANs for $s$ discrete.}

{\bf (c)} It can be shown that if we hold the encoder $\Psi$ fixed then the optimal value of the prior density $p_\Phi(z)$ is just the marginal on $z$ of the distribution defined
by sampling $s \sim \pop$ and $z \sim p_\Psi(z|s)$.  We can write this marginal as $p_{\pop,\Psi}(z)$.  Now consider the rate term when $p_\Phi(z) = p_{\pop,\Psi}(z)$.
$$\mathrm{rate} = E_{s \sim \pop,\;z \sim P_\Psi(z|s)}\;\ln\frac{p_\Psi(z|s)}{p_{\pop,\Psi}(z)}$$
Write this rate term as a differential mutual information.

\solution{
  $$\mathrm{rate} = I_{\pop,\Psi}(s,z)$$
  This has a channel capacity interpretation.  It is the information capacity (information rate) of the communication channel that takes input $y$ to output $z$.
  This is typically a nice finite number of bits (or nats) even for continuous densities.  Adding noise to $\hat{z}_\Psi(y)$ intuitively limits its precision
  and limits the information that $z$ carries about $s$.
}

\bigskip
{\bf Problem 4. 25 pts} This problem is on VAEs when both $z$ and $s$ are discrete.
Is the discreteness of $z$ an issue in this case?  Explain your answer.

\solution{Yes, the discreteness of $z$ is an issue. This is true independent of the nature of $s$. A differential change in parameters will not change a discrete $z$ and $z.\grad = 0$.
  So the standard back-propagation into the encoder fails.  VQ-VAE back-propagates into the encoder using a K-means loss term together with straight-through gradients.
  Discreteness of $s$ is not a problem.}

    
  
      
    

    

\end{document}
