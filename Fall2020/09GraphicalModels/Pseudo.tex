\input /Users/davidmcallester/ICloude/tex/SlidePreamble
\input /Users/davidmcallester/ICloude/tex/preamble

\begin{document}

{\Huge

  \centerline{\bf TTIC 31230, Fundamentals of Deep Learning}
  \bigskip
  \centerline{David McAllester, Autumn 2020}
  \vfill
  \vfill
  \centerline{\bf Pseudo-Likelihood and Contrastive Divergence}
\vfill
\vfill
\vfill

\slide{Notation}

$x$ is an input (e.g. an image).

\vfill
$\hat{\cal Y} [N]$ is a structured label for $x$ --- a vector $\hat{\cal Y}[0],\ldots,\hat{\cal Y}[N-1]$. (e.g., $n$ ranges over pixels where $\hat{\cal Y}[n]$ is a semantic label of pixel $n$.)

\vfill
$\hat{\cal Y}/n$ is the set of labels assigned by $\hat{\cal Y}$ at indeces (pixels) other than $n$.

\vfill
$\hat{\cal Y}[n=\ell]$ is the structured label identical to $\hat{\cal Y}$ except that it assigns label $\ell$ to index (pixel) $n$.

\slide{Intractable Exponential Softmax}

\vfill
We consider a softmax distribution

\vfill
\begin{eqnarray*}
P_s(\hat{\cal Y}) & = & \frac{1}{Z}e^{s(\hat{\cal Y})} \\
\\
Z & = & \sum_{\hat{\cal Y}} e^{s(\hat{\cal Y})}
\end{eqnarray*}

\vfill
Computing $Z$ is intractable.

\slide{Psuedo-Likelihood}

For any distribution $P(\hat{\cal Y})$ on structured labels $\hat{\cal Y}$,
we define the {\color{red} pseudo-likelihood}  $\tilde{P}(\hat{\cal Y})$ as follows

{\color{red} $$\tilde{P}(\hat{\cal Y}) = \prod_n\;P(\hat{\cal Y}[n]\;|\; \hat{\cal Y}/n)$$}

\vfill
$$P(\hat{\cal Y}[n]\;|\;\hat{\cal Y}/n) = \frac{1}{Z_n}e^{s(\hat{\cal Y})}\;\;\;\;\;Z_n = \sum_{\ell} e^{s(\hat{\cal Y}[n=\ell])}$$

\vfill
While computing $P_s(\hat{\cal Y})$ is intractable, computing $\tilde{P}_s(\hat{\cal Y})$ involves only local partition functions and is tractable.

\slide{Pseudo Cross-entropy Loss}

We can then do SGD on pseudo cross-entropy loss.
\vfill
{\color{red} $$\Phi^* = \argmin_\Phi E_{\tuple{x,{\cal Y}} \sim \pop}\;\;-\ln \tilde{P}_{\Phi,x}({\cal Y})$$}

\slide{Pseudolikelihood Theorem}

{\color{red} $$\argmin_Q \; E_{{\cal Y} \sim \pop} \;-\ln \tilde{Q}({\cal Y}) = \pop$$}

\vfill
It suffices to show that for any $Q$ we have

\vfill
$$ E_{{\cal Y} \sim \pop}\;-\ln \widetilde{\pop}({\cal Y}) \leq \;E_{{\cal Y} \sim \pop}\;-\ln \tilde{Q}({\cal Y})$$

\slide{Proof II}

{\huge
\begin{eqnarray*}
  & & \min_Q \;E_{{\cal Y}\sim \pop} -\ln \tilde{Q}({\cal Y}) \\
  \\
  & = & \min_Q \;E_{{\cal Y}\sim \pop} \;\sum_n\;-\ln Q({\cal Y}[n]\;|\;{\cal Y}/n) \\
  \\
  & \geq & \min_{P_1,\ldots,P_N} \;E_{{\cal Y}\sim \pop} \;\sum_n\;-\ln P_n({\cal Y}[n]\;|\;{\cal Y}/n) \\
  \\
  & = & \min_{P_1,\ldots,P_N} \;\sum_n \;E_{{\cal Y} \sim \pop}\;\; -\ln P_n({\cal Y}[n]\;|\;{\cal Y}/n) \\
  \\          
  & = & \sum_n \;\min_{P_n}\;E_{{\cal Y} \sim \pop}\;\; -\ln P_n({\cal Y}[n]\;|\;{\cal Y}/n) \\
  \\
    & = & \sum_n E_{{\cal Y} \sim \pop}\;\; -\ln \pop({\cal Y}[n]\;|\;{\cal Y}/n) =  E_{{\cal Y} \sim \pop}{-\ln \widetilde{\pop}({\cal Y})}
\end{eqnarray*}
}

\slideplain{Contrastive Divergence (CDk)}

In contrastive divergence we first construct an MCMC process whose stationary distribution is ${\color{red} P_s}$.  This could be
Metropolis or Gibbs or something else.

\vfill
{\bf Algorithm CDk}: Given a gold segmentation ${\cal Y}$, start the MCMC process from initial state ${\cal Y}$ and run the process for $k$ steps
to get ${\color{red} {\cal Y}'}$.  Then take the loss to be

\vfill
{\color{red} $${\cal L}_{\mathrm{CD}}  = s({\cal Y}') - s({\cal Y})$$}

If $P_s = \pop$ then the the distribution on ${\cal Y}'$ is the same as the distribution on ${\cal Y}$ and the
expected loss gradient is zero.

\slideplain{Gibbs CD1}

CD1 for the Gibbs MCMC process is a particularly interesting special case.

\vfill
{\bf Algorithm (Gibbs CD1)}: Given ${\cal Y}$, select a node $n$ at random and draw {\color{red} $\ell \sim P({\cal Y}[n]=\ell\;| \;{\cal Y}/n)$}. Define {\color{red} ${\cal Y}[n=\ell]$}
to be the assignment (segmentation) which is the same as ${\cal Y}$ except that node $n$ is assigned label $\ell$.  Take the loss to be

\vfill
{\color{red} $${\cal L}_{\mathrm{CD}}  = s({\cal Y}[n=\ell]) - s({\cal Y})$$}

\slide{Gibbs CD1 Theorem}

Gibbs CD1 is equivalent in expectation to pseudolikelihood.

{\huge
\begin{eqnarray*}
{\cal L}_{\mathrm{PL}} & = & E_{{\cal Y} \sim \pop}\;\sum_n \; - \ln P_s({\cal Y}\;|\;{\cal Y}/n) \\
\\
 & = & E_{{\cal Y} \sim \pop}\;\sum_n -\ln \frac{e^{s({\cal Y})}}{Z_n}\;\;\;\;\;{Z_n = \sum_{\ell'} e^{s({\cal Y}[n=\ell'])}} \\
\\
& = & E_{{\cal Y} \sim \pop}\;\sum_n\; \left(\ln Z_n - s({\cal Y})\right) \\
\\
\nabla_\Phi {\cal L}_{\mathrm{PL}} & = & E_{{\cal Y} \sim \pop}\;\sum_n \left(\frac{1}{Z_n} \sum_{\ell'} e^{s({\cal Y}[n=\ell'])} \; \nabla_\Phi\;s({\cal Y}[n=\ell'])\right) - \nabla_\Phi s({\cal Y}) \\
\\
& = & E_{{\cal Y} \sim \pop}\;\sum_n \left(\sum_{\ell'} P_s({\cal Y}[n]=\ell'\;|\;{\cal Y}/n) \; \nabla_\Phi\;s({\cal Y}[n=\ell'])\right) - \nabla_\Phi s({\cal Y})
\end{eqnarray*}
}

\slideplain{Gibbs CD1 Theorem}

{\huge
\begin{eqnarray*}
\nabla_\Phi\;{\cal L}_{\mathrm{PL}} & = & E_{{\cal Y} \sim \pop}\;\sum_n \left(\sum_{\ell'} P_s({\cal Y}[n]=\ell'\;|\;{\cal Y}/n)\; \nabla_\Phi\;s({\cal Y}[n=\ell'])\right) - \nabla_\Phi s({\cal Y}) \\
\\
& = & E_{{\cal Y} \sim \pop}\;\sum_n \left(E_{\ell' \sim P_s({\cal Y}[n]=\ell'\;|\;{\cal Y}/n)} \nabla_\Phi\;s({\cal Y}[n=\ell'])\right) - \nabla_\Phi s({\cal Y}) \\
\\
& \propto & E_{{\cal Y} \sim \pop}\;E_n\;  E_{\ell' \sim P_s({\cal Y}[n]=\ell'\;|\;{\cal Y}/n)}\;\; (\nabla_\Phi\;s({\cal Y}[n=\ell']) - \nabla_\Phi s({\cal Y})) \\
\\
& = & E_{{\cal Y} \sim \pop}\;E_n\; E_{\ell' \sim P_s({\cal Y}[n]=\ell'\;|\;{\cal Y}/n)}\;\; \nabla_\Phi\;{\cal L}_{\mbox{Gibbs CD(1)}}
\end{eqnarray*}
}

\slide{END}
}
\end{document}
