\input ../SlidePreamble
\input ../preamble

\begin{document}

{\Huge

  \centerline{\bf TTIC 31230, Fundamentals of Deep Learning}
  \bigskip
  \centerline{David McAllester, Autumn 2024}
\vfill
  \centerline{\bf Reinforcement Learning}
  \vfill
\vfill

\slide{Definition of Reinforcement Learning}

RL is defined by the following properties:

\vfill
\begin{itemize}
\item An environment with {\bf state}.

  \vfill
\item State changes are influenced by {\bf sequential decisions}.

  \vfill
\item  Reward (or loss) depends on {\bf making decisions} that lead to {\bf desirable states}.
\end{itemize}

\slide{Reinforcement Learning Examples}

\begin{itemize}
\item Board games (chess or go)

  \vfill
\item Atari Games (pong)


\vfill
\item Stock Market Investment

  \vfill
\item Robot control (driving)

\vfill
\item Dialog

  \vfill
\item Life

\end{itemize}

\slide{Agentive AI}

AI systems are becoming more ``agentive'' --- they take actions in the world.

\vfill
LLMs are currently fine tuned with reinforcement learning from human feedback (RLHF).

\vfill
As AI systems expand the range of the actions they can take AI safety becomes more of an issue.

\slidetwo{A Formal Model of Decision Making}{Markov Decision Processes (MDPs)}

An MDP consists of a set ${\cal S}$ of states, a set ${\cal A}$ of allowed actions, a reward function $R$
and a next-state probability function $P_T$.  We will use the following notation.

\vfill
$s_t \in {\cal S}$ is the state at time $t$

\vfill
$a_t \in {\cal A}$ is the action taken at time $t$.

\vfill
$r_t = R(s_t,a_t) \in \mathbb{R}$ is the reward at time $t$

\vfill
$P_T(s_{t+1}|s_t,a_t)$ is the probability of $s_{t+1}$ given $s_t$ and $a_t$.

\vfill
The function $R(s,a)$ can allow for a cost of the action $a$.

\slide{Orthogonality}

The reward $R(s,a)$ is ``orthogonal'' to the ``physics'' of the world $P_T(s_{t+1}|s_t,a)$.

\vfill
Orthogonality is often cited as a source of AI safety --- we can give the AI agents whatever motivation that we select.

\vfill
Orthogonality also seems related to identity politics --- each person their own individual sense of what is meaningful in life.

\slide{Policies}

A policy is a way of behaving.

\vfill
Formally, a (nondeterministic) policy maps a state to a probability distribution over actions.

\vfill
\begin{eqnarray*}
    \pi(a_t|s_t) & & \mbox{probability of action $a_t$ in state $s_t$ under policy $\pi$}
\end{eqnarray*}

\slide{Imitation Learning}

Construct a training set of state-action pairs $(s,a)$ from experts.

\vfill
Define stochastic policy $\pi_\Phi(s)$.

\vfill
$$\Phi^* = \argmin_\Phi \expectsub{(s,a) \sim \mathrm{Train}}{- \ln \pi_\Phi(a\;|\;s)}$$

\vfill
This is just cross-entropy loss where we think of $a$ as a ``label'' for $s$.

\slide{Dangers of Imperfect Imitation Learning}

Perfect imitation learning would reproduce expert behavior.

Imitation learning is {\bf off-policy} ---
the state distribution in the training data is different from that defined by the policy being learned.

\vfill
Imitating experts, such as expert fire eaters, can be dangersous.  ``Don't try this at home''.

\vfill
Also, it is difficult to exceed expert performance by imitating experts.

\slide{Different Ways of Aggregating Reward Over Time}

In RL we maximize reward rather than minimize loss.

$$\pi^* = \argmax_\pi \;R(\pi)$$

\vfill
$$\begin{array}{rcll}
  R(\pi) & = & E_\pi\;\sum_{t=0}^T \;r_t &\mbox{episodic reward (go)} \\
\\
& \mbox{or} & E_\pi\;\sum_{t=0}^\infty \;\gamma^t r_t &\mbox{discounted reward (financial planning)} \\
\\
& \mbox{or} & \lim_{T \rightarrow \infty}\;\frac{1}{T} \;\sum_{t=0}^T \;r_t &\mbox{asymptotic average reward (driving)}
\end{array}$$


\slide{The Value Function}

For discounted reward:

\begin{eqnarray*}
  V^\pi(s) & = & E_\pi\;\sum_t \;\gamma^t r_t  \;\;|\; \pi, \; s_0 = s \\
  \\
  V^*(s) & = & \sup_\pi \;V^\pi(s) \\
  \\
  \pi^*(a|s) & = & \argmax_a\;R(s,a) + \gamma \expectsub{s' \sim P_T(s'|s,a)}{V^*(s')} \\
  \\
  V^*(s) & = & \max_a\; R(s,a) + \gamma \expectsub{s' \sim P_T(s'|s,a)}{V^*(s')}
\end{eqnarray*}

\slide{Value Iteration}

Suppose the state space is small enough
that we can store a table $V[s]$.

\vfill
If the action space is also small we can do value iteration.

\begin{eqnarray*}
  V_0(s) & = & 0 \\
  \\
  V_{i+1}(s) & = & \max_a\; R(s,a) + \gamma \expectsub{s' \sim P_T(\cdot|s,a)}{V_i(s')}
\end{eqnarray*}

\vfill
If all rewards are non-negative then

$$V_{i+1}(s) \geq V_i(s)\;\;\;\;V_i(s) \leq V^*(s)\;\;\;\;\mathrm{so}\;\lim_{i \rightarrow \infty}\;V_i(s)\;\mathrm{exists}$$

\slide{Value Iteration}

Theorem: {\bf For discounted reward}
\vfill
$$V_\infty(s) \doteq \lim_{i \rightarrow \infty}\;V_i(s) = V^*(s)$$

\slide{Proof}

\begin{eqnarray*}
  \;\Delta & \doteq & \max_s\;\;V^*(s) - V_\infty(s) \\
  \\
  & = & \max_s \left(\begin{array}{l} \max_a R(s,a) + E_{s'|a} \gamma V^*(s') \\ - \max_a R(s,a) + E_{s'|a} \gamma V_\infty(s')\end{array}\right) \\
  \\
    & \leq & \max_s \max_a \left(\begin{array}{l} R(s,a) + E_{s'|a} \gamma V^*(s') \\ - R(s,a) + E_{s'|a} \gamma V_\infty(s')\end{array}\right) \\
  \\
  & = & \max_s \max_a \;E_{s'|a}\; \gamma (V^*(s') - V_\infty(s)) \\
  \\  
  & \leq & \gamma \Delta
\end{eqnarray*}


\slide{Summary}

\begin{itemize}
\item {\bf A Policy} $\pi$ is a stochastic way of selection an action at a state.

\vfill
\item {\bf Imitation Learning} (cross entropy imitation of action given state).

\vfill
\item Imitation Learning is {\bf off-policy}.

\vfill
\item The {\bf value function} $V^\pi(s)$.

\vfill
\item {\bf Value Iteration} $V_{i+1}(s) = \argmax_a\;R(s,a) + \gamma E_{s'}\;\gamma V_i(s')$
\end{itemize}

\slide{The Q Function}

For discounted reward:

\begin{eqnarray*}
  Q^\pi(s,a) & = & E_\pi\;\sum_t \;\gamma^tr_t \;\;|\; \pi, \; s_0 = s,\;a_0 = a \\
  \\
  Q^*(s,a) & = & \sup_\pi \;Q^\pi(s,a) \\
  \\
  \pi^*(a|s) & = & \argmax_a\;Q^*(s,a) \\
  \\
  Q^*(s,a) & = & R(s,a) + \gamma \expectsub{s' \sim P_T(\cdot|s,a)}{\max_{a'}\; Q^*(s',a')}
\end{eqnarray*}

\slide{$Q$ Function Iteration}

It is possible to define $Q$-iteration by analogy with value iteration, but this is generally not discussed.

\vfill
Value iteration is typically done for finite state spaces.  Let $S$ be the number of states and $A$ be the number of actions.

\vfill
One update of a $Q$ table takes $O(S^2A^2)$ time while one update of value iteration is $O(S^2A)$.

\slide{$Q$-Learning}

When learning by updating the $Q$ function we typically assume a parameterized $Q$ function $Q_\Phi(s,a)$.
\vfill
{\bf Bellman Error:}
{\huge $$\mathrm{Bell}_\Phi(s,a) \doteq \left(Q_\Phi(s,a) - \left(R(s,a) + \gamma\;\expectsub{s' \sim P_T(s'|s,a)}{\max_{a'}\;Q_\Phi(s',a')}\right)\right)^2$$}

{\bf Theorem}: If $\mathrm{Bell}_\Phi(s,a) = 0$ for all $(s,a)$ then the induced policy is optimal.

\vfill
{\bf Algorithm}: Generate pairs $(s,a)$ from the policy $\argmax_a\;Q_\Phi(s_t,a)$ and repeat
$$\Phi \;\minuseq\; \eta \nabla_\Phi \;\; \mathrm{Bell}_\Phi(s,a)$$


\slide{Issues with $Q$-Learning}

Problem 1: Nearby states in the same run are highly correlated.  This increases the variance of the cumulative gradient updates.

\vfill
Problem 2: SGD on Bellman error tends to be unstable. Failure of $Q_\Phi$ to model unused actions leads to policy change (exploration).
But this causes $Q_\Phi$ to stop modeling the previous actions
which causes the policy to change back ...

\vfill
To address these problems we can use a {\bf replay buffer}.

\slide{Using a Replay Buffer}

We use a replay buffer of tuples $(s_t,a_t,r_t,s_{t+1})$.

\vfill
Repeat:

\vfill
\begin{enumerate}

\item Run the policy $\argmax_a Q_\Phi(s,a)$ to add tuples to the replay buffer.  Remove oldest tuples to maintain a maximum buffer size.

\item {\color{red} $\Psi = \Phi$}
  
\item for $N$ times select a random element of the replay buffer and do
$$\Phi \;\minuseq \; \eta \nabla_\Phi\;(Q_\Phi(s_t,a_t) - (r_t + \gamma \max_{a} Q_{\color{red} \Psi}(s_{t+1},a))^2$$
\end{enumerate}

\slide{Replay is Off-Policy}

Note that the replay buffer is from a {\bf mixture of policies} and is {\bf off-policy} for $\argmax_a\;Q_\Phi(s,a)$.  This seems to be important for stability.

\vfill
This seems related to the issue of stochastic vs. deterministic policies.  More on this later.

\slide{Multi-Step $Q$-learning}

$$\Phi \;\minuseq \;\sum_t \nabla_\Phi \left(Q_\Phi(s_t,a_t) - \sum_{\delta = 0}^{D} \gamma^\delta r_{(t+\delta)}\right)^2$$

\slide{Asynchronous $Q$-Learning (Simplified)}
No replay buffer.
Many asynchronous threads each repeating:
\vfill

  \begin{quotation}
  \noindent $\tilde{\Phi} = \Phi$ (retrieve $\Phi$)\newline \newline
  \noindent using policy $\argmax_a Q_{\tilde{\Phi}}(s,a)$ compute $$s_t,a_t,r_t,\ldots,s_{t+K},a_{t+K},r_{t+K}$$
  \newline
  \noindent {\huge $\Phi \;\minuseq\; \eta \sum_{i=t}^{t+K-D} \nabla_{\tilde{\Phi}}\;\left(Q_{\tilde{\Phi}}(s_i,a_i) - \sum_{\delta = 0}^D \gamma^\delta r_{i+\delta}\right)^2$} (update $\Phi$)
  \end{quotation}

\slidetwo{Human-level control through deep RL (DQN)}{Mnih et al., Nature, 2015. (Deep Mind)}
\vfill
We consider a CNN $Q_\Phi(s,a)$.

\vfill
\centerline{\includegraphics[width=6in]{\images/DQN}}

\slide{Watch The Video}

https://www.youtube.com/watch?v=V1eYniJ0Rnk

\slidetwo{Policy Gradient Methods}
{The REINFORCE Algorithm}

\centerline{\bf Williams, 1992}

\vfill
REINFORCE is a Policy Gradient Algorithm

\vfill
We assume a parameterized policy $\pi_\Phi(a|s)$.

\vfill
$\pi_\Phi(a|s)$ is normalized while $Q_\Phi(s,a)$ is not.

\vfill
The idea is to use the stochasticity of the policy for exploration.

\slide{Policy Gradient Theorem (Episodic Case)}

$$\Phi^* = \argmax_\Phi\; E_{\pi_\Phi} \; R$$

{\huge
\begin{eqnarray*}
  \nabla_\Phi \;E_{\pi_\Phi}\;R & = & \sum_{s_0,a_0,s_1,a_1,\ldots,s_T,a_T}\;\;\nabla_\Phi P(s_0,a_0,s_1,a_1,\ldots,s_T,a_T)\;R \\
  \\
  \nabla_\Phi \;P(\ldots)R & = & P(S_0){\color{red} \nabla_\Phi \;\pi(a_0)}P(s_1)\pi(a_1) \cdots P(s_T)\pi(a_T)\;R \\
  & & + P(S_0)\pi(a_0)P(s_1){\color{red} \nabla_\Phi\;\pi(a_1)} \cdots P(s_T)\pi(a_T)\;R \\
  & & \vdots \\
  & & + P(S_0)\pi(a_0)P(s_1)\pi(a_1) \cdots P(s_T){\color{red} \nabla_\Phi\;\pi(a_T)}\;R \\
  \\
  & = & P(\ldots) \left(\sum_t\;\frac{\nabla_\Phi\;\pi_\Phi(a_t)}{\pi_\Phi(a_t)}\right) R
\end{eqnarray*}
}


\slide{Policy Gradient Theorem (Episodic Case)}
\begin{eqnarray*}
  \nabla_\Phi \;P(\ldots)R  & = & P(\ldots) \left(\sum_t\;\frac{\nabla_\Phi\;\pi_\Phi(a_t|s_t)}{\pi_\Phi(a_t|s_t)}\right) R \\
  \\
  \nabla_\Phi \;E_{\pi_\Phi}\;R & = & E_{\pi_\Phi}\;\left(\sum_t\;\nabla_\Phi\;\ln \pi_\Phi(a_t|s_t)\right) R
\end{eqnarray*}

\slide{Policy Gradient Theorem}
\begin{eqnarray*}
 & &   \nabla_\Phi \; E_{\pi_\Phi}\; R \\
  \\
  & = & E_{\pi_\Phi}\;\left(\sum_t\;\nabla_\Phi\;\ln \pi_\Phi(a_t|s_t)\right)\;\;R \\
  \\
  & = & E_{\pi_\Phi}\; \left(\sum_t\;\nabla_\Phi\;\ln \pi_\Phi(a_t|s_t)\right)\left(\sum_{t} r_{t}\right)  \\
  \\
  & = & E_{\pi_\Phi}\; \sum_{t,t'} \nabla_\Phi\;\ln \pi_\Phi(a_{t}|s_{t}) \;\;r_{t'}
\end{eqnarray*}


\slide{Policy Gradient Theorem}

\begin{eqnarray*}
    \nabla_\Phi \; E_{\pi_\Phi}\; R  & = & \sum_{t,t'}\;E_{s_t,a_t,r_{t'}}\;\; \nabla_\Phi\;\ln \pi_\Phi(a_{t}|s_{t})\;\;r_{t'}
    \end{eqnarray*}
For $t' < t$ we have
{\huge
\begin{eqnarray*}
    E_{r_{t'},s_t,a_t}\;\; r_{t'} \nabla_\Phi\;\ln \pi_\Phi(a_t|s_t)  & =  & E_{r_{t'},s_t}\; r_{t'}  \;\sum_{a_t} \; \pi_\Phi(a_t|s_t)\; \nabla_\Phi\;\ln \pi_\Phi(a_t|s_t) \\
    \\
    & =  & E_{r_{t'},s_t}\; r_{t'}  \;\sum_{a_t}\; \nabla_\Phi\; \pi_\Phi(a_t|s_t) \\
    \\
    & =  & E_{r_{t'},s_t}\; r_{t'}  \;\nabla_\Phi\; \sum_{a_t}\;\pi_\Phi(a_t|s_t) \\
    \\
    & = & 0
\end{eqnarray*}
}

\slide{REINFORCE}
$$\nabla_\Phi \;E_{\pi_\Phi}\;R \;\;\; = \;\;\; E_{\pi_\Phi}\; \sum_{t,\;t' \geq t} \; \left(\nabla_\Phi\;\ln \pi_\Phi(a_t|s_t)\right) \;r_{t'}$$

\vfill
Sampling runs and computing the above sum over $t$ and $t'$ is Williams' REINFORCE algorithm.

\slidetwo{Optimizing Discrete Decisions}{with Non-Differentiable Loss}

The REINFORCE algorithm is (or was) used generally for non-differentiable loss functions.

\vfill
For example error rate and BLEU score are non-differentiable --- they are defined on the result of discrete decisions.

\vfill
$$\Phi^* = \argmax_\Phi \;E_{w_1,\ldots,w_n \sim P_\Phi}\;\mathrm{BLEU}$$

\slide{The Variance Issue}

REINFORCE typically suffers from high variance of the gradient samples requiring very small learning rates and very long convergence times.

\begin{eqnarray*}
    \nabla_\Phi \; E_{\pi_\Phi}\; R  & = & \sum_{t,\;t'\geq t}\; E_{s_t,a_t,r_{t'}}\; \left(\nabla_\Phi\;\ln \pi_\Phi(a_{t}|s_{t})\right)\;\;r_{t'}
\end{eqnarray*}

\vfill
We will consider
\begin{itemize}
\item reducing variance due to $r_{t'}$ with Actor-Critic methods.
\item reducing variance due to  $s_t$ with Advantage Actor-Critic methods.
\item finally Asynchronous Advantge Actor-Critic Methods (A3C).
\end{itemize}

\slide{Actor-Critic Algorithms}

``Policy Gradient Methods for
Reinforcement Learning with Function
Approximation''  Mansour, Sutton, McAllester, Singh, 2000, {\color{red} 560} citations in 2018, {\color{red} 3200} citations in 2022.

\vfill
``Actor-Critic Algorithms'', Konda and Tsitsilas, 2000, {\color{red} 240} citatins in 2018, {\color{red} 4100} citations in 2022.

\vfill
These two papers both appeared at NeurIPS 2000 and are almost identical.
The first is easier to read.  The second has additional insights.


\slide{Actor-Critic Algorithms}

{\huge
\begin{eqnarray*}
    \nabla_\Phi \; E_{\pi_\Phi}\; R  & = & \sum_{t,t'}\;E_{s_t,a_t,r_{t'}}\;\nabla_\Phi\; \ln \pi_\Phi(a_{t}|s_{t})\;\;{\color{red} r_{t'}} \\
    \\
    \\
    & = & \sum_{t}\;E_{s_t,a_t}\;\nabla_\Phi\; \ln \pi_\Phi(a_{t}|s_{t})\;\;{\color{red} \sum_{t' \geq t} E_{\pi_\Phi}[r_{t'}|s_t,a_t]} \\
  \\
  \\
  & = & E_{\pi_\Phi} \; \sum_{t \geq 0}\; \left(\nabla_\Phi \;\ln \pi_\Phi(a_t|s_t)\right) \;\;{\color{red} Q^{\pi_\Phi}(s_t,a_t)} \\
  \\
  \\
 {\color{red} Q^\pi(s,a)} & = & \sum_{t\geq 0}E_\pi[r_{t}|s_0=s,,a_0=a]
 \end{eqnarray*}
}

\slideplain{An Actor-Critic Theorem}

\begin{eqnarray*}
  \nabla_\Phi \;E_{\pi_\Phi}\;R   & = & \sum_t\;E_{s_t,a_t}\; \left(\nabla_\Phi \;\ln {\color{red} \pi_\Phi(a_t|s_t)}\right) {\color{red} Q^{\pi_\Phi}(s_t,a_t)}
\end{eqnarray*}

\vfill
{\bf The point} is that we can now approximate $Q^{\pi_\Phi}$ with neural network $Q_\Theta$.

\vfill
We reduced the variance at the cost of approximating the expected future reward.


\slide{An Actor-Critic Algorithm}
\begin{eqnarray*}
   \nabla_\Phi \;E_{\pi_\Phi}\;R   & \approx & E_{\pi_\Phi} \; \sum_t\; \left(\nabla_\Phi \;\ln {\color{red} \pi_\Phi(a_t|s_t)}\right) \;\;{\color{red} Q_\Theta(s_t,a_t)}
\end{eqnarray*}

\vfill
\centerline{{\color{red} $\pi_\Phi(a|s)$} is the ``actor'' and {\color{red} $Q_\Theta(s,a)$} is the ``critic''}

\slide{An Actor-Critic Algorithm}

To get a theorem for following the loss gradient we need

{\huge
\begin{eqnarray}
  \Phi^* & = & \argmin_\Phi\left[R\;|\;\pi_\Phi\right] \label{AC1} \nn
  \nn
  \nabla_\Phi \;E_{\pi_\Phi}\;R   & = & E_{\pi_\Phi}\left[ \sum_t\; \left(\nabla_\Phi \;\ln \pi_\Phi(a_t|s_t)\right) \;\;Q_{\Theta^*(\Phi)}(s_t,a_t)\;|\; \pi_\Phi\right] \\
  \nn
  \nn
  \Theta^*(\Phi) & = & \argmin_\Theta\; E_{\pi_\Phi}\left[\sum_t\;\left(Q_\Theta(s_t,a_t) - \sum \!_{t' \geq t}\; r_{t'}\right)^2 \;|\;\pi_\Phi\right] \label{AC2}
\end{eqnarray}
}

A stationary point is now a Nash equilibrium of a two-player game defined by (\ref{AC1}) and (\ref{AC2}).


\slide{An Actor-Critic Theorem}

Thoerem:
$$\nabla_\Phi \;E_{\pi_\Phi}\; R = \sum_t\;E_{s_t,a_t} \; \left(\nabla_\Phi \;\ln \pi_\Phi(a_t|s_t)\right)(Q^{\pi_\Phi}(s_t,a_t) - V^{\pi_\Phi}(s_t))$$

\vfill
$V^{\pi_\Phi}(s) = \expectsub{a \sim \pi_\Phi(a|s)}{Q^{\pi_\Phi}(s,a)}$

\vfill
$Q^{\pi_\Phi}(s,a) - V^{\pi_\Phi}(s)$ is the ``advantage'' of deterministically using $a$ rather than sampling an action.

\slide{Proof}

We have the following for any function $V(s)$ of states.

\begin{eqnarray*}
 & & E_{s_t,a_t} \;  \; \left(\nabla_\Phi \;\ln \pi_\Phi(a_t|s_t)\right) V(s_t) \\
 \\
 & = & E_{s_t} \;  \sum_{a_t} \left(\pi_\Phi(a_t|s_t) \;\nabla_\Phi \;\ln \pi_\Phi(a_t|s_t)\right) V(s_t) \\
 \\
 & = & E_{s_t} \;  \sum_{a_t} \left(\nabla_\Phi \pi_\Phi(a_t|s_t)\right) V(s_t) \\
 \\
 & = & E_{s_t} \;  V(s_t) \nabla_\Phi \sum_{a_t}\; \pi_\Phi(a_t|s_t) \;=\; 0
\end{eqnarray*}


\slide{The Advantage Actor-Critic Algorithm}

{\huge
\begin{eqnarray}
  \Phi^* & = & \argmin_\Phi\left[R\;|\;\pi_\Phi\right] \label{ACA1} \\
  \nn
  \nabla_\Phi \;E_{\pi_\Phi}\;R   & = & E\left[ \sum_t\; \left(\nabla_\Phi \;\ln \pi_\Phi(a_t|s_t)\right) \;\;(Q_{\Theta^*(\Phi)}(s_t,a_t) - V_{\Psi^*(\Phi)}(s_t))\;|\; \pi_\Phi\right] \nn
  \nn
  \nn
  \Theta^*(\Phi) & = & \argmin_\Theta\; E\left[\sum_t\;\left(Q_\Theta(s_t,a_t) - \sum \!_{t' \geq t}\; r_{t'}\right)^2 \;|\;\pi_\Phi\right] \label{ACA2} \\
  \nn
  \nn
  \Psi^*(\Phi) & = & \argmin_\Psi\;E\left[\sum_t\;\left(V_\Psi(s_t) - Q_{\Theta^*(\Phi)}(s_t,a_t)\right)^2 \;|\;\pi_\Phi\right] \label{ACA3}
\end{eqnarray}

We now have a three player game defined by (\ref{ACA1}), (\ref{ACA2}) and (\ref{ACA3}).
}

\slide{Advantage-Actor-Critic Algorithm}
\begin{eqnarray*}
  \nabla_\Phi \;E_{\pi_\Phi}\;R   & \approx & E_{\pi_\Phi} \; \sum_t\; \left(\nabla_\Phi \;\ln \pi_\Phi(a_t|s_t)\right) \;\;(Q_\Phi(s_t,a_t) - V_\Phi(s_t))
\end{eqnarray*}

We can sample an episode and then do

\begin{eqnarray*}
  \Phi  &\pluseq & \sum_t \eta_1\left(\nabla_{\Phi}\;\ln \pi_{\Phi}(a_i|s_i)\right)(Q_\Phi(s_t,a_t) - V_\Phi(s_t))\\
  \Phi & \minuseq & \sum_t \eta_2 \;\nabla_{\Phi}\;\left(Q_\Phi(s_t,a_t) - \sum \!_{t' \geq t}\; r_{t'}\right)^2 \\
  \Phi & \minuseq & \sum_t \eta_3 \;\nabla_{\Phi}\;\left(V_\Phi(s_t) - Q_\Phi(s_t,a)\right)^2 \\
\end{eqnarray*}


\slidetwo{Asynchronous Methods for Deep RL (A3C)}{Mnih et al., Arxiv, 2016 (Deep Mind)}

\begin{quotation}
  \noindent $\tilde{\Phi} = \Phi$ (retrieve global $\Phi$)\newline
  \noindent using policy $\pi_{\tilde{\Phi}}$ compute $s_t,a_t,r_t,\ldots,s_{t+K},a_{t+K},r_{t+K}$
  $$R_i = \sum_{\delta=0}^D \gamma^{i+\delta} r_{(i+\delta)}$$
  \begin{eqnarray*}
  \Phi  &\pluseq & \eta \sum_{i=t}^{t+K-D} \left(\nabla_{\tilde{\Phi}}\;\ln \pi_{\tilde{\Phi}}(a_i|s_i)\right)\left(R_i - V_{\tilde{\Phi}}(s_i)\right)\\
  \Phi & \minuseq & \eta \sum_{i=t}^{t+K-D} \nabla_{\tilde{\Phi}}\;(V_{\tilde{\Phi}}(s_i) - R_i)^2
  \end{eqnarray*}
\end{quotation}

\slide{Issue: Policies must be Exploratory}

The optimal policy is deterministic --- $a(s) = \argmax_a Q(s,a)$.

\vfill
However, a deterministic policy never samples alternative actions.

\vfill
Typically one forces a random action some small fraction of the time.


\slide{Issue: Discounted Reward}

DQN and A3C use discounted reward on episodic or long term problems.

\vfill
Presumably this is because actions have near term consequences.

\vfill
This should be properly handled in the mathematics, perhaps in terms of the mixing time of
the Markov process defined by the policy.

\slide{Observation: Continuous Actions are Differentiable }

In problems like controlling an inverted pendulum, or robot control generally,
a continuous loss can be defined and the gradient of loss of with respect to a deterministic policy exists.

\slide{Some Videos}

https://deepmind.google/discover/blog/producing-flexible-behaviours-in-simulated-environments/

\vfill

https://www.youtube.com/watch?v=NI3NQczBWjg


\slide{END}

}
\end{document}


