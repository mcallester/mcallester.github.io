\documentclass{article}
\input ../preamble
\parindent = 0em

%\newcommand{\solution}[1]{}
\newcommand{\solution}[1]{\bigskip {\color{red} {\bf Solution}: #1}}

\begin{document}


\centerline{\bf TTIC 31230 Fundamentals of Deep Learning}
\bigskip
\centerline{\bf SGD Problems.}


\bigskip
{\bf Problem 1: Running Averages.}  Consider a sequence of vectors $x_0$, $x_1$, $\ldots$ and two running averages $y_t$ and $z_t$ defined by
as follows for $0 < \beta < 1$ and $\gamma > 0$.
\begin{eqnarray*}
  y_0 & = & 0 \\
  y_{t+1} & = & \beta y_t + (1-\beta) x_t
  \\
  \\
  z_0 & = & 0 \\
  z_{t+1} & = & \beta z_t + \gamma x_t
\end{eqnarray*}

(a) Suppose that the values $x_t$ are drawn IID from a distribution with mean vector $\overline{x} = E\;x_t$.  Give values for
$$\overline{y} = \lim_{t \rightarrow \infty} E \;y_t$$
and
$$\overline{z} = \lim_{t \rightarrow \infty} E \;z_t$$
as functions of $\beta$, $\gamma$ and $\overline{x}$

Hint: Solve for $E\;y_{t+1}$ as a function of $E\;y_t$ and assume that a limiting expectation exists.

\solution{
  \begin{eqnarray*}
   E\; y_{t+1} & = & \beta \;E\;y_t + (1-\beta) \; E\;x_t \\
   \overline{y} & = & \beta \;\overline{y} + (1-\beta)\; \overline{x} \\
   (1-\beta)\;\overline{y} & = & (1-\beta)\;\overline{x} \\
   \overline{y} & = & \overline{x}
  \end{eqnarray*}

  \begin{eqnarray*}
   E\; z_{t+1} & = & \beta \;E\;z_t + \gamma\; E\;x_t \\
   \overline{z} & = & \beta \;\overline{z} + \gamma\; \overline{x} \\
   (1-\beta)\;\overline{z} & = & \gamma\;\overline{x} \\
   \overline{z} & = & \frac{\gamma}{1-\beta}\;\overline{x}
  \end{eqnarray*}

}


(b) Express $z_t$ as a function of $y_t$, $\beta$ and $\gamma$.

\solution{
  \begin{eqnarray*}
    z_{t+1} & = & \beta \;z_t + \gamma\;x_t \\
    & = & \sum_{t'= 0}^t \gamma \beta^{t-t'} x_{t'} \\
    & = & \frac{\gamma}{1-\beta}\;\sum_{t'=0}^t (1-\beta)\beta^{t-t'} x_t \\
    & = & \frac{\gamma}{1-\beta}\;y_{t+1}
  \end{eqnarray*}
}


\bigskip
~{\bf Problem 2. Variance of an exponential moving average.}  For two independent random variables $x$ and $y$ and a weighted sum $s = ax + by$ we have
$$\sigma_s^2 = a^2\sigma_x^2 + b^2\sigma_y^2$$
Now consider a runing average for computing $\hat{\mu}_1,\ldots,\hat{\mu}_t$ from $x_1,\ldots,x_t$
$$\hat{\mu}_0 = 0$$
$$\hat{\mu}_t = \left(1-\frac{1}{N}\right)\hat{\mu}_{t-1} + \frac{1}{N}x_t$$

\medskip
(a) Assume that the values of $x_t$ are independent and identically distributed with variance $\sigma_x^2$.
We now have that $\hat{\mu}_t$ is a random variable depending on the draws of $x_t$.  The random variable $\hat{\mu}_t$ has a variance $\sigma^2_{\hat{\mu},t}$.
Assume that as $t \rightarrow \infty$ we have that $\sigma^2_{\hat{\mu},t}$ converges to a limit (it does).  Solve for this limit $\sigma^2_{\hat{\mu},\infty}$.
Your solution should yield that for $N=1$ we have $\sigma^2_{\hat{\mu},\infty} = \sigma_x^2$ (a sanity check).

\solution{
  The limit must satisfy
  $$\sigma^2_{\hat{\mu},\infty} = \left(1-\frac{1}{N}\right)^2\sigma^2_{\hat{\mu},\infty} + \frac{1}{N^2}\sigma^2_x$$
  We can then solve for $\sigma^2_{\hat{\mu},\infty}$
  \begin{eqnarray*}
    \sigma^2_{\hat{\mu},\infty} &  =  &\left(1-\frac{2}{N} + \frac{1}{N^2}\right)\sigma^2_{\hat{\mu},\infty} + \frac{1}{N^2}\sigma^2_x \\
    \\
    0 &  =  &\left(\frac{- 2}{N} + \frac{1}{N^2}\right)\sigma^2_{\hat{\mu},\infty} + \frac{1}{N^2}\sigma^2_x \\
    \\
    & = & \left((-2) + \frac{1}{N}\right)\sigma^2_{\hat{\mu},\infty} + \frac{1}{N}\sigma^2_x \\
    \\
    \sigma^2_{\hat{\mu},\infty} & = & \frac{1}{\left(2-\frac{1}{N}\right)N}\;\; \sigma_x^2
  \end{eqnarray*}
}

\medskip
(b) Compare your answer to (a) with the variance of an average of $N$ values of $x_t$ defined by
$$\hat{\mu} = \frac{1}{N}\;\sum_{t=1}^N x_t$$

\solution{
  For an average of $N$ we have $\sigma_{\hat{\mu}}^2 = \sigma_x^2/N$.  For $N$ large we have that the answer to part (a) is about half as large.
}

\bigskip
~{\bf Problem 3. Reformulating Momentum as a Exponential Moving Average.} Consider the following update equation.

\begin{eqnarray*}
  y_0  & = & 0 \\
  y_t & = & \left(1 - \frac{1}{N}\right)y_{t-1} + x_t
\end{eqnarray*}

(a) Assume that $y_t$ converges to a limit, i.e., that $\lim_{t \rightarrow \infty} y_t$ exists.
If the input sequence is constant with $x_t = c$ for all $t \geq 1$, what is $\lim_{t \rightarrow \infty}\;y_t$?  Give a derivation of your answer
(Hint: you do not need to compute a closed form solution for $y_t$).

\solution{

  The limit $y_\infty$ must satisfy
  $$y_\infty = \left(1-\frac{1}{N}\right)y_\infty + c$$
  giving $y_\infty = Nc$.
}

\medskip
(b) $y_t$ is an exponential moving average of what quantity?

\solution{
  The update can be rewritten as
  $$y_t = \left(1 - \frac{1}{N}\right)y_{t-1} + \frac{1}{N}(Nx_t)$$
  so $y_t$ is an exponential moving average of $Nx_t$.
}

\medskip
(c) Express $y_t$ as a function of $\mu_t$ where $\mu_t$ is defined by

\begin{eqnarray*}
  \mu_0  & = & 0 \\
  \mu_t & = & \left(1 - \frac{1}{N}\right)\mu_{t-1} + \frac{1}{N}x_t
\end{eqnarray*}

\solution{
  $y_t$ is an exponential moving average of $Nx_t$ which equals $N$ times the moving average of $x_t$ so we have
  $$y_t = N \mu_t$$
}

\bigskip
~{\bf Problem 4.  Bias Correction}
Consider the following update equation for computing $y_1,\ldots,y_t$ from $x_1,\ldots,x_t$.
\begin{eqnarray*}
  y_t & = & \left(1 - \frac{1}{\min(t,N)}\right)y_{t-1} + \frac{1}{\min(t,N)}\;x_t
\end{eqnarray*}

If $x_t = c$ for all $t \geq 1$ give a closed form solution for $y_t$.

\solution{
  For $t = 1$ we get $y_1 = x_1 = c$.  We then get that $y_{t+1}$ is a convex combination of $y_t$ and $x_t$ which maintains the invariant that $y_t = c$.
}

\bigskip
{\bf Problem 5.}  This problem is on interaction of learning rate and scaling of the loss function.

    \medskip

{\bf (a)} Consider vanilla SGD on cross entropy loss for classification with batch size 1 and no moment in which case we have
$$\Phi_{t+1} = \Phi_t - \eta \nabla_\Phi \ln P_\Phi(y|x)$$
Now suppose someone uses log base 2 (to get loss in bits) and uses the update
$$\Phi_{t+1} = \Phi_t - \eta' \nabla_\Phi \log_2 P_\Phi(y|x)$$
Suppose that we find that leatning rate $\eta$ works well for the natural log version (with loss in nats).
What value of $\eta'$ should be used in the second version with loss measured in bits?
You can use the relation that $\log_b z = \ln z/\ln b$.

\solution{We have
  \begin{eqnarray*}
    - \Delta \Phi & = & \eta'\nabla_\Phi \log_2 P(\Phi) \\
    & = & \eta' \nabla_\Phi \ln P(\Phi)/\ln 2 \\
    & = & \frac{\eta'}{\ln 2} \nabla_\Phi \ln P(\Phi)
  \end{eqnarray*}
  To make the two updates the same we set $\eta' = \eta \ln 2$
  }

\medskip
    
{\bf (b)} Now consider the following simplified version of RMSprop where for each parameter $\Phi[i]$ we have
$$\Phi_{t+1}[i] = \Phi_t[i] - \frac{\eta}{\sigma_i} \nabla_\Phi {\cal L}_\Phi(x_t,y_t)$$
where $\sigma_i$ is exactly the standard deviation of $i$th component of the gradient as defined by
\begin{eqnarray*}
  \mu_i & = & E_{x,y}\left[\nabla_{\Phi[i]} \;{\cal L}_\Phi(x,y) \right] \\
  \sigma_i & = & \sqrt{E_{x,y}\left[\left(\nabla_{\Phi[i]} \;{\cal L}_\Phi(x,y) - \mu_i\right)^2\right]}
\end{eqnarray*}

If we replace ${\cal L}$ by $2{\cal L}$ what learning rate $\eta'$ should we use with loss $2{\cal L}$ to get the same temperature?

\solution{If we double the loss function we also double $\sigma_i$ and we have $\eta' = \eta$.  For RMSprop we get that the learning rate is (approximately) invariant
to scaling the loss function.  It is not clear whether this has any significance.}


\bigskip
{\bf Problem 6. Adaptive SGD.}  This problem considers the question of whether the convergence theorem hold for adaptive methods ---
in the limit as the learning rate goes to zero do adaptive methods converge to a local minimum of the loss.

Consider a generalization of RMSProp where we allow an arbitrary adaptation with with different learning rates for
different parameter values.  More specifically consider the SGD update equation

$$(1)\;\;\;\;\Phi_{t+1} = \Phi_t - \eta\left(A(\Phi_t,x_t,y_t)\odot \nabla_\Phi {\cal L}(\Phi_t,x_t,y_t)\right)$$

where $\tuple{x_t,y_t}$ is the $t$th training pair, $A(\Phi_t,x_t,y_t)$ is an adaptation vector, and $\odot$ is the Haddamard product $(x \odot y)[i] = x[i]\;y[i]$.

Consider the special case given by
\begin{eqnarray*}
  A(\Phi,x,y)[i] & = & \frac{1}{\sqrt{s(\Phi,x,y)} + \epsilon} \\
  \\
  s(\Phi,x,y) & = & \frac{1}{d} ||\nabla_\Phi\;{\cal L}(\Phi,x,y)||^2 \\
\end{eqnarray*}
where $d$ is the dimension of $\Phi$.
\medskip

(a) For the given interpretation of $A(\Phi,x,y)$, let $\Phi^*$ be a parameter setting that is a stationary point of the update equation (1)
in the sense that expected update over a random draw from the population is zero.  Write this stationary condition
on $\Phi^*$ explicitly as an expectation equaling zero under the given interpretation of $A(\Phi,x,y)$.

\solution{
$$E_{\tuple{x,y} \sim \pop}\;\frac{1}{\sqrt{s(\Phi^*,x,y)+ \epsilon}}\;\nabla_\Phi\;{\cal L}(\Phi,x,y) = 0$$
}

\medskip
(b) Is $\Phi^*$ as defined in part (a) a stationary point of the original loss --- a point where the expected gradient of ${\cal L}(\Phi^*,x,y)$ is equal to zero?

\solution{
  No, the average a weighted sum is different from the average of an unweighted sum
  and hence the fact that the weighted average is zero does not imply that the average is zero.
  }
  
\medskip
(c) Do these observations have implications for the adaptive methods described in this class.  Explain your answer.

\solution{Yes, the example considered here is just a special case of RMSProp or Adam which are in fact not guaranteed to converge to a stationary point (or local optimum) of the loss function.}

\bigskip
{\bf Problem 7}  This problem is on a non-standard form of adaptive learning rates.  In general when we consider the significance of a change $\Delta x$ to a number $x$ it is reasonable to consider
the change as a percentage of $x$.  For example, a baseline annual raise in salary is often a percentage raise when different employees have significantly different salaries.  So we might consider the following
``multiplicative update SGD'' which we will write here for batch size 1.

\begin{equation}
  \Phi^{t+1}[i] = \Phi^t[i] - \eta \;\max(\epsilon,|\Phi^t[i]|)\;\;\hat{g}(\Phi,x_t,y_t)[i]
  \label{mult}
\end{equation}

where $\hat{g}(\Phi,x,y)$ abbreviates the gradient $\nabla_\Phi{\cal L}(\Phi,x,y)$ where ${\cal L}(\Phi,x,y)$ is the loss for the training point $(x,y)$ at parameter setting $\Phi$, and where
and $\hat{g}(\Phi,x,y)[i]$ is the $i$th component of the gradient.  For $|\Phi^t[i]| >> \epsilon$ this is a multiplicative update.
Multiplicative updates have a long history and rich theory for mixtures of experts prior to the deep revolution.  However, I do not know of a citation for
the above multiplicative variant of SGD (let me know if you find one later).  The parameter $\epsilon$ allows a weight to flip sign --- to pass through zero more easily.

Recall that a stationary point is a parameter setting where the total gradient is zero.

\begin{equation}
  \sum_{(x,y) \sim \train}\; \nabla_\Phi\;{\cal L}(x,y) = 0
  \label{stationary}
\end{equation}

\medskip
    {\bf (a)} At a stationary point of the loss function, is the expected update of equation (\ref{mult}) over a random draw of $(x_t,y_t)$ always equal to zero.  In other words, is a stationary point of the loss function
    also a stationary point of the update equation?

\solution{Yes, a stationary point of the loss function is also a stationary point of the update equation.
  \begin{eqnarray*}
    & & E_{(x,y) \sim \train} \;\;\eta\;\max(\epsilon,|\Phi^t[i]|)\;\left(\nabla_\Phi \;{\cal L}(\Phi,x,y)\right)[i]  \\
    \\
    & = & \eta\;\max(\epsilon,|\Phi[i]|)\;E_{(x,y)\sim\train} \left(\nabla_\Phi {\cal L}(\Phi,x,y)\right)[i] \\
    \\
    & = & 0
  \end{eqnarray*}
  }

\medskip
{\bf (b)} Consider an adaptive algorithm which makes the update proportional to the loss. i.e.,
\begin{equation}
  \Phi^{t+1}= \Phi^t - \eta\;{\cal L}(\Phi,x_t,y_t)\;\hat{g}^t
  \label{loss}
\end{equation}
Is a stationary point of the loss function always a stationary point of the update defined by (\ref{loss})?  Justify your answer.

\medskip
You can assume that there exists a training set of two points $(x_1,y_1)$ and $(x_2,y_2)$ and a stationary point of the loss
function $\Phi$ with ${\cal L}(\Phi,x_1,y_1) \not =  {\cal L}(\Phi,x_2,y_2)$ and $\nabla_\Phi(\Phi,x_1,y_1) \not = \nabla_\Phi(\Phi,x_2,y_2)$.

\solution{No, the expected update can be non-zero at a stationary point of the loss function.  Weighing the updates by something that depends on the draw of $(x,y)$ effectively changes the weighting on the training points
  which changes the stationarity condition.  Writing this in English counts as a correct solution. A formal counter example can be given using the assumed conditions:
\begin{eqnarray*}
  & & E_{(x,y) \sim \train} \;\;\eta\;{\cal L}(\Phi,x,y)\;\;\nabla_\Phi \;{\cal L}(\Phi,x,y)  \\
  \\
  & = & \eta \; \frac{1}{2}\left({\cal L}(\Phi,x_1,y_1)\;\left(\nabla_\Phi \;{\cal L}(\Phi,x_1,y_1)\right) + {\cal L}(\Phi,x_2,y_2)\;\left(\nabla_\Phi \;{\cal L}(\Phi,x_2,y_2)\right)\right) \\
  \\
  & = & \eta \; \frac{1}{2}\left({\cal L}_1\;\left(\nabla_\Phi \;{\cal L}(\Phi,x_2,y_2)\right) + {\cal L}_2\;\left(\nabla_\Phi \;{\cal L}(\Phi,x_2,y_2)\right)\right) \\
  \\
  & = & \eta ({\cal L}_1 + {\cal L}_2) \; \frac{1}{2}\left(\frac{{\cal L}_1}{{\cal L}_1 + {\cal L}_2}\;\left(\nabla_\Phi \;{\cal L}(\Phi,x_2,y_2)\right) + \frac{{\cal L}_2}{{\cal L}_1 + {\cal L}_2}\;\left(\nabla_\Phi \;{\cal L}(\Phi,x_2,y_2)\right)\right) \\
    \\
    & \not = & \eta \;({\cal L}_1 + {\cal L}_2) \frac{1}{2}\left(\;\nabla_\Phi \;{\cal L}(\Phi,x_1,y_1) + \nabla_\Phi \;{\cal L}(\Phi,x_2,y_2)\right) \\
    \\
    & = & 0
  \end{eqnarray*}

In Adam and RMSProp we have a weighting that depends on a moving average of the second moment of the gradients.  This is essentially a weighting that depends on a random draw over the training data.
It has been shown that stationary points of Adam and RMSProp updates do not necessarily correspond to stationary points of the loss function.
}

\end{document}
