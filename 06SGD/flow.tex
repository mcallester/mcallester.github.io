\input ../../SlidePreamble
\input ../../preamble

\begin{document}

{\Huge
  \centerline{\bf TTIC 31230, Fundamentals of Deep Learning}
  \bigskip
  \centerline{David McAllester, Winter 2020}
  \vfill
  \centerline{\bf Stochastic Gradient Descent (SGD)}
  \vfill
  \vfill
  \centerline{\bf Gradient Flow}
  \vfill

\slide{Gradient Flow}

Gradient flow is a non-stochastic ({\color{red} deterministic}) model of {\color{red} stochastic} gradient descent (SGD).

\vfill
Gradient flow is defined by the {\color{red} total gradient} differential equation

$${\color{red} \frac{d \Phi}{d t} = - g(\Phi) \;\;\;\;\;\; g(\Phi) = \nabla_\Phi\; E_{(x,y) \sim \mathrm{Train}}\;{\cal L}(\Phi,x,y)}$$

\vfill
We let $\Phi(t)$ be the solution to this differential equation satisfying $\Phi(0) = \Phi_{\mathrm{init}}$.

\slide{Gradient Flow}
$${\color{red} \frac{d \Phi}{d t} = - g(\Phi)}$$

\vfill
For small values of $\Delta t$ this differential equation can be approximated by

\vfill
$${\color{red} \Delta \Phi = - g(\Phi)\Delta t}$$

\slide{Time as the Sum of the Learning Rates}

Consider the update.


{\color{red} $$\Delta \Phi = - g\Delta t$$}

\vfill
Here $\Delta t$ has both a natural interpretation as time in a numerical simulation of the flow differential equation.

\vfill
But it also has a natural interpretation as a learning rate.

\vfill
This leads to interpreting the sum of the learning rates as ``time'' in SGD.

\slide{Gradient Flow and SGD}
Consider a sequence of model parameters $\Phi_1$, $\ldots$, $\Phi_N$ produced by SGD with
$$\Phi_{i+1} = \Phi_i - \eta \hat{g}_i$$
and where $\hat{g}_i$ is the gradient of the $i$th randomly selected training point.

\vfill
Take $\eta \rightarrow 0$ and $N \rightarrow \infty$ using $N = t/\eta$.  We will show that in this limit for SGD we have that $\Phi_N$ converges to $\Phi(t)$
as defined by gradient flow.

\slide{Gradient Flow and SGD}
For $\Phi_{i+1} = \Phi_i - \eta \hat{g}_i$ we divide $\Phi_1$, $\ldots$, $\Phi_N$ into $\sqrt{N}$ blocks.
$$(\Phi_1,\dots,\Phi_{\sqrt{N}})\;(\Phi_{\sqrt{N}+1},\ldots,\Phi_{2\sqrt{N}})\;\cdots\;(\Phi_{T-\sqrt{N}+1},\ldots,\Phi_N)$$

\vfill
For $\eta \rightarrow 0$ and $N = t/\eta$ we have $\eta\sqrt{N} \rightarrow 0$ which implies
$$\Phi_{\sqrt{N}} \sim \Phi_0 - \eta\sqrt{N} g$$
where $g$ is the average (non-stochastic)
gradient.

\vfill
Since the gradients within each block become non-stochastic, we are back to gradient flow.


\slide{END}
\end{document}
