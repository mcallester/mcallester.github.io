\input ../SlidePreamble
\input ../preamble

\newcommand{\solution}[1]{\bigskip {\bf Solution}: #1}

\begin{document}

{\Huge
  \centerline{\bf TTIC 31230, Fundamentals of Deep Learning}
  \bigskip
  \centerline{David McAllester, Autumn 2020}
  \vfill
  \centerline{\bf Stochastic Gradient Descent (SGD)}
  \vfill
  \centerline{\bf RMSProp and Adam}

\slide{RMSProp and Adam}

RMSProp and Adam are ``adaptive'' SGD methods --- they use different learning rates for different model parameters where the parameter-specific learning rate is
computed from statistics of the data.

\vfill
Adam is variant of RMSProp with momentum and ``debiasing''.

\slide{RMSProp}

RMSProp was introduced in Hinton's class lecture slides.

\vfill
RMSProp is based on a running average of $\hat{g}[i]^2$ for each scalar model parameter $i$.

\begin{eqnarray*}
{\color{red} s_t[i]} & {\color{red} =} & {\color{red} \left(1-\frac{1}{N_s}\right) s_{t-1}[i] + \frac{1}{N_s} \hat{g}_t[i]^2}\;\;\;\mbox{$N_s$ typically 100 or 1000} \\
\\
{\color{red} \Phi_{t+1}[i]} & {\color{red} =} & {\color{red} \Phi_t[i] - \frac{\eta}{\sqrt{s_t[i]} + \epsilon}\;\; \hat{g}_t[i]}
\end{eqnarray*}

\slide{RMSProp}
The second moment of a scalar random variable $x$ is $E\;x^2$

\vfill
The variance $\sigma^2$ of $x$ is $E \;(x - \mu)^2$ with $\mu = E\;x$.

\vfill
RMSProp uses an estimate $s[i]$ of the second moment of the random scalar $\hat{g}[i]$.

\vfill
If the mean $g[i]$ is small then $s[i]$ approximates the variance of $\hat{g}[i]$.

\vfill
There is a ``centering'' option in PyTorch RMSProp that switches from the second moment to the variance.

\slide{RMSProp Analysis}

For sufficiently small $\epsilon$, RMSProp makes the update independent of the scale of the gradient.

{\color{red} $$\Phi[i] \;\minuseq \; \eta\;\frac{\hat{g}[i]}{\sqrt{s[i]}}$$}

If the gradient $\hat{g}_i$ is scaled up by a factor of $\alpha$ we have that $\sqrt{s[i]}$ is also scaled by $\alpha$.

\vfill
This makes the learning rate $\eta$ dimensionless.

\slide{RMSProp Analysis}

{\color{red} $$\Phi[i] \;\minuseq \; \eta\;\frac{\hat{g}[i]}{\sqrt{s[i]}}$$}

\vfill
It is not completely clear why making the update invariant to gradient scaling is a good thing.

\vfill
It would be nice to have an analysis directly tied to convergence rates.

\slide{Adam --- Adaptive Momentum}

Adam combines momentum and RMSProp.

\vfill
PyTorch RMSProp also supports momentum.  However, it presumably uses the standard momentum learning rate parameter which couples the temperature to both
the learning rate and the momentum parameter.  Without an understanding of the coupling to temperature, hyper-parameter optimization is then difficult.

\vfill
Adam uses a momentum parameter that is naturally decoupled from temperature.

\vfill
Adam also uses ``bias correction''.

\slide{Bias Correction}

Consider a standard moving average.

\begin{eqnarray*}
\tilde{x}_0 & = & 0 \\
\\
\tilde{x}_t & = & \left(1-\frac{1}{N}\right)\tilde{x}_{t-1} + \left(\frac{1}{N}\right)x_t
\end{eqnarray*}

\vfill
For $t < N$ the average $\tilde{x}_t$ will be strongly biased toward zero.

\slide{Bias Correction}

The following running average maintains the invariant that $\tilde{x}_t$ is exactly the average of $x_1,\ldots,x_t$.

\begin{eqnarray*}
\tilde{x}_t & = & \left(\frac{t-1}{t}\right)\tilde{x}_{t-1} + \left(\frac{1}{t}\right)x_t \\
\\
\\
& = & \left(1-\frac{1}{t}\right)\tilde{x}_{t-1} + \left(\frac{1}{t}\right)x_t
\end{eqnarray*}

\vfill
We now have $\tilde{x}_1 = x_1$ independent of any $x_0$.

\vfill
But this fails to track a moving average for $t >> N$.

\slide{Bias Correction}

The following avoids the initial bias toward zero while still tracking a moving average.

\begin{eqnarray*}
\tilde{x}_t & = & \left(1-\frac{1}{\min(N,t)}\right)\tilde{x}_{t-1} + \left(\frac{1}{\min(N,t)}\right)x_t
\end{eqnarray*}

\vfill
The published version of Adam has a more obscure form of bias correction which yields essentially the same effect.

\slide{Adam (simplified)}

\begin{eqnarray*}
  \tilde{g}_{t}[i] & = & \left(1-\frac{1}{\min(t,N_g)}\right)\tilde{g}_{t-1}[i] + \frac{1}{\min(t,N_g)} \hat{g}_t[i] \\
  \\
  \\
  s_{t}[i] & = & \left(1-\frac{1}{\min(t,N_s)}\right)s_{t-1}[i] + \frac{1}{\min(t,N_s)} \hat{g}_t[i]^2 \\
  \\
  \\
\Phi_{t+1}[i] & =  & \Phi_t[i] - \frac{\eta}{\sqrt{s_{t}[i]} + \epsilon}\;\;\tilde{g}_{t}[i]
\end{eqnarray*}

\slide{Decoupling $\eta$ from $\epsilon$}

$$\Phi_{t+1}[i] =  \Phi_t - \frac{\eta}{\sqrt{s_{t}[i]} + \epsilon}\;\;\tilde{g}_{t}[i]$$

\vfill
The optimal $\epsilon$ is sometimes large.
For large $\epsilon$ it is useful to set $\eta = \epsilon\eta_0$ in which case we get

$$\Phi_{t+1}[i] =  \Phi_t - \frac{\eta_0}{1 + \frac{1}{\epsilon}\sqrt{s_{t}[i]}}\;\;\tilde{g}_{t}[i]$$

\vfill
We then get standard SGD  as $\epsilon \rightarrow \infty$ holding $\eta_0$ fixed.

\slide{Making Adam Adapt to the Batch Size $B$}

Adam alreaady adapts to momentum by using and EMA of the gradient as momentum.
Adapting to batch size requires some analysis.

\vfill
$$\Phi[i] \;\minuseq \; \eta\;\frac{\hat{g}[i]}{\sqrt{s[i]}}$$

\vfill
Since we are taking a long moving average of $\hat{g}[i]^2$ we can assume $s[i] = E\;\hat{g}[i]^2$ and we have

      \begin{eqnarray*}
        s[i] & = & E \;\hat{g}[i]^2 \;\; = \;\; \mu^2 + \sigma^2/B \\
 \mu & = & E\;\hat{g}[i] \; = \; E\;g[i]\\
 \sigma^2 & = & E (g[i] - \mu)^2
  \end{eqnarray*}  

\slide{Making Adam Adapt to the Batch Size $B$}

\vfill
\begin{eqnarray*}
\Phi[i] & \minuseq & \eta\;\frac{\hat{g}[i]}{\sqrt{\mu^2 + \sigma^2/B}} \\
\\
E\;\Delta\Phi & = & - \eta \frac{\mu}{\sqrt{\mu^2 + \sigma^2/B}} \\
\end{eqnarray*}

For $\mu^2 >> \sigma^2/B$ we want $\eta = B \eta_0$ as with vanilla SGD.

\vfill
For $\sigma^2/B >> \mu^2$ we want $\eta = \sqrt{B} \;\eta_0$.


\slide{Making Adam Independent of $B$}

The previous analysis looses information.  We could estmate $g[i]^2$ more accurately by computing $g[i]^2$ for each batch element
rather than measuring $\hat{g}[i]^2$ where $\hat{g}[i]$ is alread averaged over the batch.

\vfill
This would require adding a batch index
to the parameter gradients.

\slide{END}

} \end{document}
