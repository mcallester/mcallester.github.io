\input ../../SlidePreamble
\input ../../preamble

\begin{document}

{\Huge
  
  \centerline{\bf TTIC 31230, Fundamentals of Deep Learning}
  \bigskip
  \centerline{David McAllester, Winter 2020}
  \vfill
  \vfill
  \centerline{\bf CNNs: Einstein Notation}
  \vfill
  \vfill
  \vfill


\slide{Einstein Notation}

For the representation of general relativity, Einstein introduced the convention of explicitly writing all indeces of tensors where repeated indeces in a product of tensors are implicitly summed.

\vfill
Writing indeces explicitly improves the clarity of the notation at the expense of not being in correspondence with framework notation.  Most frameworks hide indeces.

\vfill
This course will focus on conceptual understanding rather than framework implementations.  For conceptual understanding Einstein notation seems preferable.


\slide{Einstein Notation}

Einstein notation improves tensor equations for tensors with many indeces.

\vfill
However, we start by considering just vectors and matrices.

\vfill
We will use a modified form of Einstein notation where captial letters are used to denote slices of a tensor.  For example:

\vfill
\begin{itemize}
\item $M[i,j]$ denotes one element of the matrix $M$.
\item $M[i,J]$ denotes the $i$th row of $M$.
\item $M[I,j]$ denotes the $j$th collumn of $M$.
\item $M[I,J]$ denotes the full matrix $M$.
\end{itemize}

\slide{Einstein Notation}
Repeated capital letters in a product of tensors denote summation over those letters.

\vfill
\begin{eqnarray*}
y = Wx &\;\;\;\equiv\;\;\; & y[i] = \sum_j \;W[i,j]x[j] \\
& \;\;\;\equiv \;\;\; & y[i] = W[i,J]x[J] \\
\\
\\
y = x^\top\;W & \;\;\;\equiv \;\;\; & y[j] = \sum_i \;W[i,j]x[i] \\
& \;\;\;\equiv \;\;\; & y[j] = W[I,j]x[I]
\end{eqnarray*}

\ignore{

\slide{An MLP in Einstein Notation}

\begin{eqnarray*}
  {\color{red} h[j]} & = & \sigma\left(W^0[j,I] \;{\color{red} x[I]} - b^0[j]\right) \\
  \\
  {\color{red} s[\hat{y}]} & = & \sigma\left(W^1[\hat{y},J]\;{\color{red} h[J]} - b^1[\hat{y}]\right) \\
  \\
  {\color{red} P_\Phi[\hat{y}]} & = & \softmax_{\hat{y}}\;{\color{red} s[\hat{y}]}
\end{eqnarray*}
}

\slide{Einstein Notation for Convolution}

CNNs provide a good example of the advantage of Einstein Notation.

\vfill
{\color{red} $L[b,x,y,i]$} is the value of ``neuron'' $i$ for batch element $b$ at image position $\tuple{x,y}$.

\slide{Convolution}
\centerline{\includegraphics[width = 2.5in]{\images/Convolution}}
\centerline{$W[\Delta x,\Delta y,i,j]$\hspace{6ex}$L_{{\ell}}[b,x,y,i]$\hspace{6ex}$L_{{\ell+1}}[b,x,y,j]$}
\centerline{\large River Trail Documentation}

{\huge
\begin{eqnarray*}
L_{{\ell+1}}[b,x,y,j] & = & \sigma\left(\sum_{\Delta x,\Delta y,i}\;W[\Delta x, \Delta y, i,j]\; L_{{\ell}}[b,x + \Delta x, y + \Delta y, i] - B[j]\right) \\
\\
\\
& = & \sigma\left(W[\Delta X, \Delta Y, I,j]\; L_{{\ell}}[b,x + \Delta X, y + \Delta Y, I] - B[j]\right)
\end{eqnarray*}
}
\slide{Types and Einstein Notation}

The indeces of tensors generally have types such as a ``time index'', ``x coordinate'', ``y coordinate'', ``batch index'', or ``neuron index''.

\vfill
Writing a matrix as $W[T,I]$ where $T$ is a time index and $I$ is a feature index makes the type of the matrix $W$ clear and clarifies
the order of the indeces (disambiguates $W$ from $W^\top$).

\vfill
Writing a layer of a CNN as $L[B,X,Y,I]$ clarifies both the types and the positions of the four indeces.

\slide{END}
}

\end{document}
