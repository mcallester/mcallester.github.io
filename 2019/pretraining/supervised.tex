\input ../SlidePreamble
\input ../preamble


\begin{document}

{\Huge
  \centerline{\bf TTIC 31230,  Fundamentals of Deep Learning}
  \vfill
  \centerline{David McAllester, Winter 2020}
  \vfill
  \centerline{\bf Supervised Imagenet Pretraining}
  \vfill
  \vfill
  
\slide{Supervised ImageNet Pretraining}

CBNet: A Novel Composite Backbone Network Architecture for Object Detection
Liu et al., Sept. 2019 (COCO leader as of February 26, 2020).

\vfill
\begin{quotation}
Generally speaking, in a typical CNN based object detector, a backbone network is used to extract basic features for detecting objects, which is usually designed for
the image classification task and pretrained on the ImageNet
dataset.
\end{quotation}

\slide{Instagram Pretraining, Mahajan et al., May 2018}

In our experiments, we train standard convolutional network architectures to
predict hashtags on up to 3.5 billion public Instagram images.

\vfill
To make training at this scale practical, we adopt a distributed synchronous implementation of
stochastic gradient descent with large (8k image) minibatches, following Goyal et al. 2017.

\slide{Instagram Pretraining}

\centerline{\includegraphics[width=10.0 in]{../images/InstagramPre}}

\slide{Rethinking ImageNet Pretraining, He et al., Nov. 2018}

We report competitive results on object detection and instance segmentation on the COCO dataset using standard
models trained from {\bf random initialization}.


\slide{Rethinking ImageNet Pretraining, He et al., Nov. 2018}

\centerline{\includegraphics[width=6.0 in]{../images/RethinkingPre}}

\slide{CBNet (COCO leader as of Feb, 2020)}

\centerline{\includegraphics[height=5.2 in]{\images/CBNet}}


\slide{CBNet (COCO leader as of Feb, 2020)}

\begin{quotation}
We initialize each assembled backbone of CBNet with the pretrained model of the single backbone which is widely and
freely available today, such as ResNet and ResNeXt
\end{quotation}

\slide{END}

}
\end{document}
