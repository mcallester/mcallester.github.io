\input ../SlidePreamble
\input ../preamble

\begin{document}

{\Huge
  
  \centerline{\bf TTIC 31230, Fundamentals of Deep Learning}
  \bigskip
  \centerline{David McAllester, Winter 2020}
  \vfill
  \vfill
  \centerline{\bf Backpropagation for Scalar Source Code}
  \vfill
  \vfill

\slide{\bf Backpropagation (backprop)}

Backpropagation is the method frameworks use to compute $\nabla_\Phi\;{\cal L}_\Phi(z)$
for source code ${\cal L}_\Phi(z)$.

\slide{Some Simple Source Code}

The expression

\vfill
{\color{red} $${\cal L} = \sqrt{x^2 + y^2}$$}

\vfill
can be transformed to the assignment sequence

{\color{red}
\vfill
\begin{eqnarray*}
  u & = & x^2  \\
  v & = & y^2 \\
  r & =& u + v \\
  {\cal L} & = & \sqrt{r}
\end{eqnarray*}
}

\anaslide{Source Code}
\vspace{-1ex}
{\color{red}
$$\begin{array}{lcl}
 1.\;u & = & x^2  \\
 2.\;w & = & y^2 \\
 3.\;r & =& u + w \\
  4.\;{\cal L} & = & \sqrt{r}
\end{array}$$
}

\vfill
For each variable $z$, the derivative $\partial {\cal L}/\partial z$ will get computed in reverse order.

\vfill
{\color{red}
$$\begin{array}{lcl}
(4)\; \partial{\cal L}/\partial r & = & \frac{1}{2\sqrt{r}} \\
(3)\; \partial{\cal L}/\partial u & = & \partial {\cal L}/\partial r \\
(3)\; \partial{\cal L}/\partial w & = & \partial {\cal L}/\partial r\\
(2)\; \partial{\cal L}/\partial y & = &  (2y) * (\partial {\cal L}/\partial w) \\
(1)\; \partial{\cal L}/\partial x & = & (2x) * (\partial {\cal L}/\partial u)
\end{array}$$
}

\anaslideplain{A More Abstract Example (Still Scalar Values)}
\vspace{-3ex}
\begin{eqnarray*}
  y & = & f(x) \\
  z & = & g(y,x) \\
  u & = & h(z) \\
  {\cal L} & = & u
\end{eqnarray*}

\medskip
For now assume all values are scalars (single numbers rather than arrays).

\medskip
We will ``backpopagate'' the assignments the reverse order.

\anaslide{Backpropagation (Scalar Values)}
\vspace{-3ex}
\begin{eqnarray*}
  y & = & f(x) \\
  z & = & g(y,x) \\
  u & = & h(z) \\
  {\cal L} &  = & {\color{red} u}
\end{eqnarray*}

\medskip
{\color{red} ${\partial {\cal L}}/{\partial u} = 1$}

\anaslide{Backpropagation (Scalar Values)}
\vspace{-3ex}
\begin{eqnarray*}
  y & = & f(x) \\
  z & = & g(y,x) \\
  u & = & h({\color{red} z}) \\
  {\cal L} &  = &  u
\end{eqnarray*}

\medskip
${\partial {\cal L}}/{\partial u} = 1$

\medskip
{\color{red} ${\partial {\cal L}}/{\partial z} = ({\partial {\cal L}}/{\partial u}) ({\partial u}/{\partial z})\;$} (this uses the value of $z$)

\anaslide{Backpropagation (Scalar Values)}
\vspace{-3ex}
\begin{eqnarray*}
  y & = & f(x) \\
  z & = & g({\color{red} y},x) \\
  u & = & h(z) \\
  {\cal L} &  = &  u
\end{eqnarray*}

\medskip
${\partial {\cal L}}/{\partial u} = 1$

\medskip
{\color{red} ${\partial {\cal L}}/{\partial z} = ({\partial {\cal L}}/{\partial u}) ({\partial u}/{\partial z})\;$} (this uses the value of $z$)

\medskip
{\color{red} ${\partial {\cal L}}/{\partial y} = ({\partial {\cal L}}/{\partial z}) ({\partial z}/{\partial y})$} (this uses the value of $y$ and $x$)

\anaslide{Backpropagation (Scalar Values)}
\vspace{-3ex}
\begin{eqnarray*}
  y & = & f({\color{red} x}) \\
  z & = & g(y,{\color{red} x}) \\
  u & = & h(z) \\
  {\cal L} &  = &  u
\end{eqnarray*}

\medskip
${\partial {\cal L}}/{\partial u} = 1$

\medskip
{\color{red} ${\partial {\cal L}}/{\partial z} = ({\partial {\cal L}}/{\partial u}) ({\partial u}/{\partial z})\;$} (this uses the value of $z$)

\medskip
{\color{red} ${\partial {\cal L}}/{\partial y} = ({\partial {\cal L}}/{\partial z}) ({\partial z}/{\partial y})$} (this uses the value of $y$ and $x$)

\medskip
{\color{red} ${\partial {\cal L}}/{\partial x} =$ ???} Oops, we need to add up multiple occurrences.

\anaslide{Backpropagation (Scalar Values)}
\vspace{-3ex}
\begin{eqnarray*}
  y & = & f({\color{red} x}) \\
  z & = & g(y,{\color{red} x}) \\
  u & = & h(z) \\
  {\cal L} &  = &  u
\end{eqnarray*}

\medskip
Each framework program variable denotes an {\color{red} object} (in the sense of C++ or Python).

\medskip
{\color{red} $x.\mathrm{value}$} and {\color{red} $x.\grad$} are attributes of the {\color{red} object $x$}.

\bigskip
Values are computed ``forward'' while gradients are computed ``backward''.


\anaslide{Backpropagation (Scalar Values)}
\vspace{-3ex}
\begin{eqnarray*}
  y & = & f(x) \\
  z & = & g(y,x) \\
  u & = & h(z) \\
  {\color{red} {\cal L}} &  {\color{red} =} & {\color{red}  u} \\
  {\color{red} z.\grad} & {\color{red} =} & {\color{red} y.\grad = x.\grad = 0} \\
  {\color{red} u.\grad} & {\color{red} =} & {\color{red} 1} \\
\end{eqnarray*}

\vfill
{\bf Invariant}: The gradients are correct for the red program.

\anaslide{Backpropagation (Scalar Values)}
\vspace{-3ex}
\begin{eqnarray*}
  y & = & f(x) \\
  z & = & g(y,x) \\
  {\color{red} u} & {\color{red} =} & {\color{red} h(z)} \\
  {\color{red} {\cal L}} &  {\color{red} =} & {\color{red}  u} \\
  {\color{red} z.\grad} & {\color{red} =} & {\color{red} y.\grad = x.\grad = 0} \\
  {\color{red} u.\grad} & {\color{red} =} & {\color{red} 1} \\
  {\color{red} z.\grad} & {\color{red} \pluseq} & {\color{red} u.\grad * ({\partial u}/{\partial z})}
\end{eqnarray*}

\vfill
{\bf Invariant}: The gradients are correct for the red program.

\anaslide{Backpropagation (Scalar Values)}
\vspace{-3ex}
\begin{eqnarray*}
  y & = & f(x) \\
  {\color{red} z} & {\color{red} =} & {\color{red} g(y,x)} \\ 
  {\color{red} u} & {\color{red} =} & {\color{red} h(z)} \\
  {\color{red} {\cal L}} &  {\color{red} =} & {\color{red}  u} \\
  {\color{red} z.\grad} & {\color{red} =} & {\color{red} y.\grad = x.\grad = 0} \\
  {\color{red} u.\grad} & {\color{red} =} & {\color{red} 1} \\
  {\color{red} z.\grad} & {\color{red} \pluseq} & {\color{red} u.\grad * ({\partial u}/{\partial z})} \\
  {\color{red} y.\grad} & {\color{red} \pluseq} & {\color{red} z.\grad * ({\partial z}/{\partial y})} \\
  {\color{red} x.\grad} & {\color{red} \pluseq} & {\color{red} z.\grad * ({\partial z}/{\partial x})}
\end{eqnarray*}


\anaslide{Backpropagation (Scalar Values)}
\vspace{-3ex}
\begin{eqnarray*}
  {\color{red} y} & {\color{red} =} & {\color{red} f(x)} \\
  {\color{red} z} & {\color{red} =} & {\color{red} g(y,x)} \\ 
  {\color{red} u} & {\color{red} =} & {\color{red} h(z)} \\
  {\color{red} {\cal L}} &  {\color{red} =} & {\color{red}  u} \\
  {\color{red} z.\grad} & {\color{red} =} & {\color{red} y.\grad = x.\grad = 0} \\
  {\color{red} u.\grad} & {\color{red} =} & {\color{red} 1} \\
  {\color{red} z.\grad} & {\color{red} \pluseq} & {\color{red} u.\grad * ({\partial u}/{\partial z})} \\
  {\color{red} y.\grad} & {\color{red} \pluseq} & {\color{red} z.\grad * ({\partial z}/{\partial y})} \\
  {\color{red} x.\grad} & {\color{red} \pluseq} & {\color{red} z.\grad * ({\partial z}/{\partial x})} \\
  {\color{red} x.\grad} & {\color{red} \pluseq} & {\color{red} y.\grad * ({\partial y}/{\partial x})}
\end{eqnarray*}

\slide{END}
}

\end{document}
                
