\input ../SlidePreamble
\input ../preamble

\begin{document}

{\Huge
  
  \centerline{\bf TTIC 31230, Fundamentals of Deep Learning}
  \bigskip
  \centerline{David McAllester, Winter 2020}
  \vfill
  \vfill
  \centerline{\bf Minibatching: The Batch Index}
  \vfill
  \vfill

 \slide{Minibatching}
 
We run some number of instances together (or in parallel) and then do a parameter update based on the average
gradients of the instances of the batch.

\vfill
For NumPy minibatching is not so much about parallelism as about making the vector operations larger so that the vector operations dominate
the slowness of Python.  On a GPU minibatching allows parallelism over the batch elements.
\vfill

\slide{Minibatching}
\vfill
With minibatching each input value and each computed value is actually a batch of values.

\vfill
We add a batch index as an additional first tensor dimension for each input and computed node.

\vfill
Parameters do not have a batch index.

\slide{The Batch Index}

We let $B$ be the number of elements in a single minibatch and let $b$, with $0 \leq b \leq B-1$, be an index naming a particular element of the current minibatch.

\slide{MLP with a Batch Index}

\centerline{$b$--- batch index,\hspace{3em} $i$ --- input feature index}
\centerline{$j$ --- hidden layer index, \hspace{3em} $\hat{y}$ --- possible label}
$$\Phi = (W^0[j,i],\;b^0[j],\;W^1[\hat{y},j],\;b^1[\hat{y}])$$

\vfill
\begin{eqnarray*}
  {\color{red} h[b,j]} & = & \sigma\left(\sum_i \left(W^0[j,i] \;{\color{red} x[b,i]}\right) - b^0[j]\right) \\
  \\
  {\color{red} s[b,\hat{y}]} & = & \sigma\left(\sum_j \left(W^1[\hat{y},j]\;{\color{red} h[b,j]}\right) - b^1[\hat{y}]\right) \\
  \\
  {\color{red} P_\Phi[b,\hat{y}]} & = & \softmax_{\hat{y}}\;{\color{red} s[b,\hat{y}]}
\end{eqnarray*}

\slideplain{Backpropagation with Minibatching}
\vspace{-3ex}
\begin{eqnarray*}
  \mathrm{for}\;b,i,j\;\;\;\tilde{h}[b,j] & \pluseq & W[j,i]\;x[b,i]
\end{eqnarray*}

\begin{eqnarray*}
\mathrm{for}\;b,i,j\;\;\;  x.\mathrm{grad}[b,i] & \pluseq & W[j,i]\tilde{h}.\mathrm{grad}[b,j] \\
  \\
\mathrm{for}\;b,i,j\;\;  W.\mathrm{grad}[j,i] & \pluseq & {\color{red} \frac{1}{B}}\;x[b,i]\tilde{h}.\mathrm{grad}[b,j]
\end{eqnarray*}

\vfill
$B$ is the number of batch elements.  By convention parameter gradients are averaged over the batch.

\slide{Setting the Batch Size}

The batch is typically made as large as the hardware will support.

\vfill
Theoretically extremely large batches can make SGD require more epochs and hence more energy consumption even for parallel batch processing.

\vfill
But empirically, unless one has thousands of GPUs, it seems the batch can be as large as possible without requiring additional epochs.
\slide{END}
}

\end{document}
